\section{Introduction} \label{sec:introduction}

The raising of social networks as worldwide means of communication and expression, is gaining lot of interest from company and academia, due to the huge availability of daily contents published by users.
Focusing on academia perspective, especially in the Natural Language Processing field, the  contents available in form of written text are really useful for the study of specific open problems, where the stance detection related to political events is an example, and the \emph{Stance and Gender Detection in Tweets on Catalan Independence (StanceCat)} task at IberEval 2017 is a concrete application.

In \emph{StanceCat}, the principal aim is to automatically detect if the text's author is in favor of, against, or neutral towards the Catalan Independence. Moreover, as a secondary aim, participants are asked to infer the author's gender. 

To tackle the described problem we built a \emph{stance}\&\emph{gender detection} system mainly decomposed in two modules: text pre-processing and classification model.
During the system's tuning process, different design choices were explored trying to find the best modules' combination and from their anlysis some interesting insight can be drawn.

In the following sections we firstly describe the StanceCat task (\Cref{sec:task}), then we illustrate the module's design of developed stance\&gender detection system (\Cref{sec:system}), after that, an evaluation of the tuning process for submitted systems is analysed (\Cref{sec:evaluation}), finally, conclusion over the whole work are outlined (\Cref{sec:conclusion}).
