\section{Task definition} \label{sec:task}


\emph{Maite PRSOCO example}

The main objective proposed by the organizers of the PRSOCO shared task was to predict the personality traits of developers given a collection of their source code. The personality of a developer was determined following the Five Factor Theory or Big Five [5, 11, 3] which is the most widely accepted in psychology. Therefore, five traits define the personality of an author. Those traits are: agreeableness (A), conscientiousness (C), extroversion (E), openness to experience (O), and emotional stability / neuroticism (N). Each trait was labeled within a range between 20 and 80. The models were evaluated by the organizers using two metrics: the average Root Mean Squared Error (RMSE) as well as the Pearson Product-Moment Correlation (PC). For further information about the task, please review the overview paper of the task [16].


\emph{COSET webpage example}

Political conversation in Twitter increases when a General Election comes close. Analyzing the topics discussed by users provides interesting insights of this growing public conversation on politics.

In COSET, the aim is to classify a corpus of political tweets in 5 categories of classification: political issues, related to the most abstract electoral confrontation; policy issues, about sectorial policies; personal issues, on the life and activities of the candidates; campaign issues, related with the evolution of the campaign; and other issues.

The tweets are written in Spanish and they talk about the 2015 Spanish General Election. In the training phase participants will be provided with Twitter Ids and their manually issue codification.

\emph{TODO }

problem over the tweet (chiedi alla tipa di torino i tweet challenges)

\emph{First draft}

The aim proposed by organizers of COSET shared task was to classify tweets from Spanish election in five different categories:
\begin{enumerate*}
\item political issues, related to the most abstract electoral confrontation; 
\item policy issues, about sectorial policies; 
\item personal issues, on the life and activities of the candidates; 
\item campaign issues, related with the evolution of the campaign;
\item and other issues.
\end{enumerate*}

 the personality traits of developers given a collection of their source code. The personality of a developer was determined following the Five Factor Theory or Big Five [5, 11, 3] which is the most widely accepted in psychology. Therefore, five traits define the personality of an author. Those traits are: agreeableness (A), conscientiousness (C), extroversion (E), openness to experience (O), and emotional stability / neuroticism (N). Each trait was labeled within a range between 20 and 80. The models were evaluated by the organizers using two metrics: the average Root Mean Squared Error (RMSE) as well as the Pearson Product-Moment Correlation (PC). For further information about the task, please review the overview paper of the task [16].
 
\subsection{Copora statistics}

Average train sequence length: 135 chars (in chars)

Average train sequence length: 24 (in words)

Max train sequence length: 49 (words)

si è ottimizzato il parametro del numero di parole da usare come input alla rete neurale, arrivando alla misura di 30

numeri superiore risultano in un eccesso di padding e perdita di informazione della rete. Numeri inferiori perdono troppa informazione

