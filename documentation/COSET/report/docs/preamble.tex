% This file contains your LaTeX preamble. A preamble is a part of your document where all required packages and macros can be defined. This needs to be done before the \begin{document} command.

% Documentclass:
% Standard LaTeX classes are: article, book, report, slides, and letter. These cover the basis, but are not best. More advanced users might want to try out the KOMA classes or the memoir class. Optional arguments: 10pt. The font size of the main content is set to 10pt with the option between [].
\documentclass[10pt]{llncs}

% Geometry:
% The papersize of the document is defined with the geometry package. Here, the size is set to A4 with a4paper. Other possibilities are a5paper, b5paper, letterpaper, legalpaper and executivepaper.
\usepackage[a4paper]{geometry}

% AMS math packages:
% Required for proper math display.
\usepackage{amsmath,amsfonts,amsthm}
%\usepackage{amsfont} create a conflict so it's not used
\usepackage{amssymb}
\usepackage{mathtools}
% For case equation with nested label
% \usepackage{cases} NOT USEFUL



% Graphicx:
% If you want to include graphics in your document, the graphicx package is required.
\usepackage{graphicx}

% Graphic path declaration:
\graphicspath{{figs/}}

% Tabu:
% To test if this package could be the latest package for my work
\usepackage{tabu}

% [OLD] Booktabs:
% The booktabs package is needed for better looking tables. 
\usepackage{booktabs}

% [OLD] Tables:
\usepackage[table]{xcolor}
\usepackage{multirow}
%\usepackage{tabularx}

\newcommand{\win}{\cellcolor[gray]{0.9}}

% Enumeration:
% To manipulate the listing of object
\usepackage[inline]{enumitem}
\setlist*[enumerate,1]{%
	label=(\roman*),
}
% [Alternative]
%\usepackage{easylist}

% SIunitx:
% The SIunitx package enables the \SI{}{} command. It provides an easy way of working with (SI) units.
\usepackage{siunitx}

% URL:
% Clickable URL's can be made with this package: \url{}.
%\usepackage{url}

% Caption:
% For better looking captions. See caption documentation on how to change the format of the captions.
\usepackage{caption}

% Hyperref:
% This package makes all references within your document clickable. By default, these references will become boxed and colored. This is turned back to normal with the \hypersetup command below.
\usepackage{hyperref}
\hypersetup{colorlinks=false,pdfborder=0 0 0}

% Cleveref:
% This package automatically detects the type of reference (equation, table, etc.) when the \cref{} command is used. It then adds a word in front of the reference, i.e. Fig. in front of a reference to a figure. With the \crefname{}{}{} command, these words may be changed.
\usepackage{cleveref}
\crefname{equation}{equation}{equations}
\crefname{figure}{figure}{figures}	
\crefname{table}{table}{tables}

% Pseudo code packages:
\usepackage[ruled,linesnumbered]{algorithm2e}
\usepackage{algorithmicx}
\usepackage{algpseudocode}

% Verbatim:
% Typewriter formatting and comment environment
\usepackage{verbatim}

% Bibliography style
%\usepackage{natbib}



% Command

% For the level structure depending of the 