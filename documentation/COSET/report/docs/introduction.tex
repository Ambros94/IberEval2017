\section{Introduction} \label{sec:introduction}

Nowadays the pervasive use of social network as a mean of communication helps researchers to found useful insight over open problems in the field of Natural Language Processing. %(put the cite of other work that use social network).
In this context, the \emph{Twitter} social network has a huge role in text classification problems, because thanks to its \emph{API} is possible to retrieve specific formatted text (i.e. a sentence of maximum 140 characters called tweet) from a huge real-time text database, where different users publish their daily statements.

This huge availability of data gives raise to the investigation of new text classification problems, with special interest in prediction problems related to temporal events that can influence statements published by social network users. An example of this problem category is the text classification related to general election, where the  \emph{Classification Of Spanish Election Tweets (COSET)} task at IberEval 2017 is a concrete example. % try to find some citing about it

In COSET, the aim is to classify a corpus of political tweets in five categories related to specific political topics. This task can be analysed as a domain-dependent (i.e. political domain) constrained-text (i.e. tweet sentence) classification problem.

To tackle the above problem we built a classification system that can be decomposed in three main modules, each representing a specific approach widely used in the NLP literature: text pre-processing, text representation and classification model.
During the modules design, we explore different design combinations leading the system development to a comparative study over the possible modules interactions.
Analysing the produced study interesting insight can be drawn to create a system baseline for the tweet classification problem.

In the following sections we firstly describe the COSET task (\Cref{sec:task}), then we report the development process of the classification system and its module design (\Cref{sec:system}), after that, the evaluation of deployed systems over the provided corpus is analysed (\Cref{sec:evaluation}), finally, conclusion over the whole work are outlined (\Cref{sec:conclusion}).

% TO DO in future
% handcrafted features
% explain that we haven't put a proper module for handcrafted features, but our modular approach to solve the classification problem lead to an easy customization and integration of a proper module for these handcrafted features.


