\section{Evaluation} \label{sec:evaluation}

In this section we are going to illustrate results of the comparative study drawn during the system development, first we illustrate the metric used to evaluate the system (\cref{subsec:metric}) and then we report the results produced by a 10-fold cross validation over the development data set (\cref{subsec:results}).

\subsection{Metrics} \label{subsec:metric}

System evaluation metric were given by organizers and reported here in the following \cref{eq:f1macro,eq:f1,eq:precision,eq:recall}, their choice was to use an $F_{1-macro}$ measure due to class unbalance in the corpus.

\begin{multicols}{2}
\begin{equation} \label{eq:f1macro}
F_{1-macro} = \frac{1}{|L|} \displaystyle\sum_{l\in L} F_1(y_l, \hat{y}_l)
\end{equation}

\begin{equation} \label{eq:f1}
F_1 = 2 \cdot \frac{precision \cdot recall }{precision + recall}
\end{equation}

\begin{equation} \label{eq:precision}
precision = \frac{1}{|L|} \displaystyle\sum_{l\in L} Pr(y_l, \hat{y}_l)
\end{equation}

\begin{equation} \label{eq:recall}
recall = \frac{1}{|L|} \displaystyle\sum_{l\in L} R(y_l, \hat{y}_l)
\end{equation}
\end{multicols}



\subsection{Results} \label{subsec:results}

In the continuation we present a comparative study over possible combination of pre-processing (\cref{tab:preprocessing}) and text representation (\cref{tab:representation}), in both case results are calculated from averaging three runs of a 10-fold cross validation over the development set (\textbf{LUCA put in your word the explanation of training/development set evaluation}).
Notations used in \cref{tab:preprocessing} refers to the one introduced in \cref{subsec:preprocessing} (the listing of a notation means its use for the reported result, where SW and PR were merged as SR), regarding the tweet specific pre-processing all the items have been substituted, exception for the URL and RW that have been removed (in the table reported as RM). From all the possible combinations (i.e. $2^3\cdot3^{7}+1$) we report only the most relevant ones.

\begin{table}[h]
\footnotesize
\caption{Pre-processing study comparing 10-fold cross validation results over the development set in terms of percentuage of $F_{1-macro}$ score. }
\label{tab:preprocessing}
\centering
\begin{tabular}{l|ccccc}
\toprule
\multirow{2}{*}{Preprocessing}	& \multicolumn{5}{ c }{Models}       \\ 
						& CNN		& LSTM		& B-LSTM	& FAST-TEXT	& KIM	\\ 
\hline 
Nothing					& 50,5     & 55,6		& 51,1		& 53,0 		& \textbf{55,8}	\\ 
\hline 
ST						& 49,6		& 49,9		& 47,5		& \textbf{53,1}	& \textbf{53,1}	\\ 
ST+SR					& 47,6		& \textbf{55,3}	& 47,6		& 52,9		& 51,1	\\ 
ST+SR+RM				& 51,9		& \win\textbf{56,8}		& 52,2		& 56,0		& 50,8	\\ 
ST+SR+RM+MT				& 54,6		& \textbf{56,1}		& 47,7		& 54,6		& 53,6	\\ 
ST+SR+RM+MT+NUM			& 53,2		& \textbf{55,5}		& 51,0		& 53,4		& 51,5 \\ 
ST+SR+RM+MT+NUM+EM+SM	& 52,7		& \textbf{56,4}		& 53,7		& 54,2		& 51,8 \\ 
ST+SR+RM+MT+NUM+EM+SM+HT& 55,1		& 54,0		& 52,9		& \textbf{56,7}		& 51,1 \\ 
\hline
SR+RM+MT+NUM+EM+SM		& 54,5		& 54,8		& 53,9		& \win\textbf{57,0}		& 54,8 \\ 
\hline
RM						& 55,9		& 43,4		& 48,5		& 56,5		& \textbf{57,7} \\ 
RM+EM+SM				& \win57,1		& 52,1		& 49,9		& 54,8		& \win\textbf{58,9} \\
RM+MT+NUM+EM+SM			& 54,3		& 54,6		& \win55,5		& \textbf{56,8}		& 54,8 \\ 
\bottomrule
\end{tabular}
\end{table}

conclusion over pre-processing
\begin{itemize}
\item CL serve a tutti (URL e RW rumore)
\item recurrent model perform better with ST+SW, while other peggiorano
\item fast text just cl improve performance a lot
\end{itemize}


Intro su text representation

\begin{table}[h]
\footnotesize
\caption{Text representation study comparing 10-fold cross validation results over the development set in terms of percentuage of $F_{1-macro}$ score. The pre-processing setting was fixed at RM+EM.}
\label{tab:representation}
\centering
\begin{tabular}{l|ccccc}
\toprule
\multirow{2}{*}{Embedding}	& \multicolumn{5}{ c }{Models}       \\ 
					& CNN		& LSTM		& B-LSTM	& FAST-TEXT	& KIM	\\ 
\hline 
ES static			& \textbf{48,1}		& 36,1		& 38,9		& 36,4		& 43,6\\
ES non-static		& \win57,1		& 52,1		& 49,9		& \win54,8		& \win\textbf{58,9}\\
\hline
CA static			& \textbf{45,1}		& 30,6		& 38,5		& 30,1		& 39,6\\ 
CA non-static		& 53,5		& \win53,6		& 46,3		& 54,1		& \textbf{56,2}\\
\hline
Online				& 52,6		& 47,3		& \win51,8		& 53,3		& \textbf{54,7}\\
\bottomrule
\end{tabular}
\end{table}

conclusion over text representation, introduction over the use of catal pre-trained vector

\begin{itemize}
\item static always worse, CMC  learnt over wikipedia
\item no-static always improve
\item if no-static + pretrained improve respect just no-static, even if used a pre-trained of a similar language
\item best result when non-static es
\end{itemize}