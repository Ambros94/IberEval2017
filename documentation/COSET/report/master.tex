% This is the master file of the folder structure.

% First, the preamble needs to be called. This contains all the 'under the hood' stuff for your document.
% Use \input rather than than \include for .tex files, because \input can be nested and don't include a page break.
% This file contains your LaTeX preamble. A preamble is a part of your document where all required packages and macros can be defined. This needs to be done before the \begin{document} command.

% Documentclass:
% Standard LaTeX classes are: article, book, report, slides, and letter. These cover the basis, but are not best. More advanced users might want to try out the KOMA classes or the memoir class. Optional arguments: 10pt. The font size of the main content is set to 10pt with the option between [].
%\documentclass[xcolor=table]{beamer}
\documentclass[xcolor=table,handout]{beamer}

\mode<presentation> {
	
	% The Beamer class comes with a number of default slide themes
	% which change the colors and layouts of slides. Below this is a list
	% of all the themes, uncomment each in turn to see what they look like.
	
	%\usetheme{default}
	%\usetheme{AnnArbor}
	%\usetheme{Antibes}
	%\usetheme{Bergen}
	%\usetheme{Berkeley}
	%\usetheme{Berlin}
	%\usetheme{Boadilla}
	%\usetheme{CambridgeUS}
	%\usetheme{Copenhagen}
	%\usetheme{Darmstadt}
	%\usetheme{Dresden}
	%\usetheme{Frankfurt}
	%\usetheme{Goettingen}
	%\usetheme{Hannover}
	%\usetheme{Ilmenau}
	%\usetheme{JuanLesPins}
	%\usetheme{Luebeck}
	\usetheme{Madrid}
	%\usetheme{Malmoe}
	%\usetheme{Marburg}
	%\usetheme{Montpellier}
	%\usetheme{PaloAlto}
	%\usetheme{Pittsburgh}
	%\usetheme{Rochester}
	%\usetheme{Singapore}
	%\usetheme{Szeged}
	%\usetheme{Warsaw}
	
	% As well as themes, the Beamer class has a number of color themes
	% for any slide theme. Uncomment each of these in turn to see how it
	% changes the colors of your current slide theme.
	
	%\usecolortheme{albatross}
	%\usecolortheme{beaver}
	%\usecolortheme{beetle}
	%\usecolortheme{crane}
	%\usecolortheme{dolphin}
	%\usecolortheme{dove}
	%\usecolortheme{fly}
	%\usecolortheme{lily}
	%\usecolortheme{orchid}
	%\usecolortheme{rose}
	%\usecolortheme{seagull}
	%\usecolortheme{seahorse}
	%\usecolortheme{whale}
	%\usecolortheme{wolverine}
	
	%\setbeamertemplate{footline} % To remove the footer line in all slides uncomment this line
	%\setbeamertemplate{footline}[page number] % To replace the footer line in all slides with a simple slide count uncomment this line
	
	%\setbeamertemplate{navigation symbols}{} % To remove the navigation symbols from the bottom of all slides uncomment this line
}

\usepackage{graphicx} % Allows including images
\usepackage{booktabs} % Allows the use of \toprule, \midrule and \bottomrule in tables

% ADDED BY MYSELF
\graphicspath{{figs/}}
\usepackage{tikz} % for the logo management

%Pseudo code package
\usepackage[ruled,linesnumbered]{algorithm2e}
\usepackage{algorithmicx}
\usepackage{algpseudocode}

% Table
%\usepackage[table]{xcolor}
\usepackage{tabularx}
\usepackage{multirow}

% Diagram
\usepackage{tikz}
\usepackage{verbatim}
%\usepackage[margin=15mm]{geometry}
\usetikzlibrary{shapes,arrows,fit,calc,positioning}

% Algorithm
%\usepackage[ruled,dotocloa,algochapter]{algorithm2e}
%\usepackage[boxruled,vlined,linesnumbered]{algorithm2e}
%\usepackage[linesnumbered]{algorithmicx}
%\usepackage{algpseudocode}
\usepackage[]{algorithm2e}
%\resetcounteronoverlays{algocf}

% algorithm environment
\newenvironment{agreement}[1][htb]
{\renewcommand{\algorithmcfname}{Agreement}% Update algorithm name
	\begin{algorithm}[#1]%
	}{\end{algorithm}}

% References
\usepackage{cleveref}

% Font 
\usepackage{lmodern}	% textttt plus bold
\usepackage{bm}			% math plus bold with \bm

% Footnotes
%\usepackage[perpage]{footmisc}	% reset the footnote counter after each slide

% Slides command
\AtBeginSection[]
{
	\begin{frame}
	\frametitle{Table of Contents}
	\tableofcontents[currentsection]
\end{frame}
}

\AtBeginSubsection[]
{
\begin{frame}
\frametitle{Table of Contents}
\tableofcontents[currentsection,currentsubsection]
\end{frame}
}


% SCRIPT

% table color
\newcommand{\win}{\cellcolor[gray]{0.9}}


% Copyright
\newcommand{\cmark}{$^{\tiny{\textcopyright}}$}

% Registered trademark 
\newcommand{\rmark}{$^{\tiny{\textregistered}}$}

\newcommand{\bfttt}[1]{\textbf{\texttt{#1}}}

% The title page is created with the command \maketitle which needs to be placed after the \begin{document} command. To create the titlepage, some entries are needed: the name of the autor is defined by \author{}, the title by the entry \title{} and the date by the command \date{}. Note that the current date is displayed with \today.
\author{Luca Ambrosini \and Giancarlo Nicol\`{o}}
\title{ALC}
\date{}

%\titlerunning{<Your abbreviated contribution title>}
%\authorrunning{<abbreviated author list>}
\institute{Univesitat Polit\`{e}cnica De Val\`{e}ncia}
%	\and <name of the next institute>}

% All the actual content of your document should be placed after \begin{document} and before \end{document}. This content should be placed in the docs folder and can then be called with \input{docs/filename}.
\begin{document}

% Here the actual title page is printed, based on the given entries \author{}, \title{} and \date{}.
\maketitle

% The table of contents can be automatically generated with the \tableofcontents command. Note that you need to compile the document twice in order to see the changes in the table of contents.
%\tableofcontents

% List of figures
%\listoffigures
 
% List of tables
%\listoftables

% The list of algorithm
%\listofalgorithms

% The \input{} command reads and processes the indicated example.tex file. Note that docs/ locates the folder where the .tex file is stored.
\abstract

Version one

This paper describes an explorative approach to create from scratch a set of baselines for tackling tweet classification problems using natural language processing and machine learning techniques.
Our approach focuses on neural network models taken from state of the art that exploit different corpus preprocessing, tweets representations and features extraction methods.
Finally, we will discuss how we applied this methodology to the \emph{Classification Of Spanish Election Tweets (COSET)} task at IberEval 2017 and present the results we obtained

Version two

This paper describes our participation in the \emph{Classification Of Spanish Election Tweets (COSET)} task at IberEval 2017.
During the searching process for the best classification system to send for the competition, we developed a comparative study over possible combination of corpus preprocessing, text representations and classification models. After an initial models exploration, we focus our attention over specific neural models.
Interesting insight can be drawn from the comparative study helping future practitioner tackling tweets classification problems to create system baseline for their work.

Possible add for the report in general and 

handcrafted features

explain that we haven't put a proper module for handcrafted features, but our modular approach to solve the classification problem lead to an easy customization and integration of a proper module for these handcrafted features.		% Abstract
\section{Introduction} \label{sec:introduction}

Intro over nlp and text classification: motivation why this field is important and the actual challenge and open problems (will be top just to list some problem and their way to be solved, for example citing the different workshop that we had during the course)


Text classification is an important task in Natural Language Processing with many applications, such as web search, information retrieval, ranking and document classification (Deerwester et al., 1990; Pang and Lee, 2008). Recently, models based on neural networks have become increasingly popular (Kim, 2014; Zhang and LeCun, 2015; Conneau et al., 2016). While these models achieve very good performance in practice, they tend to be relatively slow both at train and test time, limiting their use on very large datasets.

The 

vedi citazioni su word embedding di mark e bag of word da maite

Introduce informally our task and list some way of tackling this problem from the different perspectives, maybe say that we de-construct tthe actual approach and cathegorized their inner process in: representation, preprocessing, model and post processing (that we haven't done).

Finally explain how we structured this report

\subsection{INFO}

Average train sequence length: 135 chars (in chars)

Average train sequence length: 24 (in words)

Max train sequence length: 49 (words)

si è ottimizzato il parametro del numero di parole da usare come input alla rete neurale, arrivando alla misura di 30

numeri superiore risultano in un eccesso di padding e perdita di informazione della rete. Numeri inferiori perdono troppa informazione
	% Introduction
\section{Task definition} \label{sec:task}

The StanceCat shared task aim was to detect the author's gender and stance with respect to the target \emph{independence of Catalonia} in tweets written in Spanish and Catalan, where participants are allowed to detect both stance and gender or only stance \cite{taule2017overview}.

Participants had access to a labelled corpus of 4319 tweets for each language. We analysed it and find the following statistical informations presented in \cref{tab:tweet,tab:label}.

\begin{comment}


\begin{table}[h]
\footnotesize
\centering
\begin{tabular}{l|cccc}
\toprule
\hline
Label		& Favor		& Neutral		& Against	& Total		\\
\hline
ES			& 335		& 2538			& 1446		& 4319		\\
CA			& 2648		& 1540			& 131		& 4319		\\
\hline
\bottomrule
\end{tabular}
\caption{Statistical analysis of tweets' label from given corpus.}
\label{tab:labelold}
\end{table}

\begin{table}[h]
\footnotesize
\centering
\begin{tabular}{l|ccc}
\toprule
\hline
Tweets		& Average		& Deviation		& Max		\\
\hline
ES			& 14			& 3				& 23		\\ 
CA			& 13			& 4				& 20		\\ 
\hline
\bottomrule
\end{tabular}
\caption{Statistical analysis of given corpus' tweets regarding words length.}
\label{tab:tweetold}
\end{table}

\end{comment}

\begin{table}[h]
	\centering
	\caption{Statistical analysis of tweets' label from given corpus.}
	\label{tab:label}
	\begin{tabular}{l|cccc}
		\hline\noalign{\smallskip}
		Label		& Favor		& Neutral		& Against	& Total		\\
		\noalign{\smallskip}
		\hline
		\noalign{\smallskip}
		ES			& 335		& 2538			& 1446		& 4319		\\
		CA			& 2648		& 1540			& 131		& 4319		\\
		\hline
	\end{tabular}
\end{table}


\begin{table}[h]
	\centering
	\caption{Statistical analysis of given corpus' tweets regarding words length.}
	\label{tab:tweet}
	\begin{tabular}{l|cccc}
		\hline\noalign{\smallskip}
		Tweets		& Average		& Deviation		& Max		\\
		\noalign{\smallskip}
		\hline
		\noalign{\smallskip}
		ES			& 14			& 3				& 23		\\ 
		CA			& 13			& 4				& 20		\\ 
		\hline
	\end{tabular}
\end{table}

			% Task description
\section{Systems description} \label{sec:system}

In this section we describe the stance\&gender detection systems. Organizing the system by modules, it can be divided in two blocks: text pre-preprocessing (\Cref{subsec:preprocessing}) and classification model (\Cref{subsec:classificationModel}). 


\subsection{Text pre-processing} \label{subsec:preprocessing}
Regarding the text pre-preprocessing, has to be mentioned that the corpus under observation can not be treated as proper written language, because computer-mediated communication (CMC) is highly informal, affecting diamesic\footnote{The variation in a language across medium of communication (e.g. Spanish over the phone versus Spanish over email)} variation with creation of new items supposed to pertain lexicon and graphematic domains \cite{bazzanella2011oscillazioni,cerruti2013netspeak}.
Therefore, our pre-processing follows two approaches: classic and microblogging related.
As classic aproach we used stemming (i.e., ST), stopwords (i.e., SW) and punctuation removal (i.e., PR).
For microblogging approach we focus our attention over the following items:
\begin{enumerate*}
\item mentions (i.e., MT),
\item smiley (i.e., SM),
\item emoji (i.e., EM),
\item hashtags (i.e., HT),
\item numbers (i.e., NUM),
\item URL (i.e., URL)
\item and Tweeter reserve-word as RT and FAV (i.e., RW).
\end{enumerate*}
For each of these items we leave the possibility to be removed or substituted by constant string.
In relation to above approaches we implement them using the following tools:
\begin{enumerate*}
\item NLTK \cite{nltk} and 
\item Preprocessor\footnote{Preprocessor is a preprocessing library for tweet data written in Python, https://github.com/s/preprocessor}.
\end{enumerate*}



\subsection{Classification models} \label{subsec:classificationModel}
Following, we describe the neural models used for the classification module. Before introducing the models we describe the specific text representation used as input layer \Cref{subsubsec:representation} (i.e., sentence-matrix).

\subsubsection{Text representation} \label{subsubsec:representation}
To represent the text we used word embeddings as described by \cite{bojanowski2016enriching}, where \emph{words} are represented as vectors of real number with fixed dimension $|v|$.
In this way a whole sentence $s$, with length $|s|$ its number of word, is represented as a \emph{sentence-matrix} $M$ of dimension $|M| = |s| \times |v|$. $|M|$ has to be fixed a priori, therefore $|s|$ and $|v|$ have to be estimated. $|v|$ was fixed to 300 following \cite{bojanowski2016enriching}. $|s|$ was estimated analyzing \cref{tab:tweet}, in details we decided to fix it as the sum of average length plus the standard deviation (i.e. $|s| = 17$ for both language), with this choice input sentences longer than $|s|$ are truncated, while shorter ones are padded with null vectors (i.e., a vector of all zeros).

Choosing words as elements to be mapped by the embedding function, raise some challenge over the function estimation related to data availability. In our case the available corpus is very small and estimated embeddings could lead to low performance.
To solve this problem, we decided to used a pre-trained embeddings estimated over Wikipedia using a particular approach called \emph{fastText} \cite{bojanowski2016enriching}.


\subsubsection{Convolutional Neural Network.}
Convolutional Neural Networks (CNN) are considered state of the art in many text classification problem. Therefore, we decide to use them in a simple architecture composed by a convolutional layer, followed by a \emph{Global Max Pooling} layer and two dense layers.

\subsubsection{Dilated KIM.}
This model is our new topology of CNN. It can be seen as an extension of Kim's model \cite{kim2014convolutional} using the dilation ideas from computer graphics field \cite{yu2015multi}.

The original Kim's model is a particular CNN where the convolutional layer has multiple filter widths and feature maps.
The complete architecture is illustrated in \Cref{fig:kim}, here the input layer (i.e., sentence-matrix) is processed in a convolutional layer of multiple filters with different width, each of these results are fed into \emph{Max Pooling} layers and finally the concatenation of them (previously flatten to be dimensional coherent) is projected into a dense layer.
Our extension is to use a dilated filters in combination with normal ones, the intuition is that normal filter capture \emph{adjacent words} features, while dilated one are able to capture relations between \emph{non adjacent words}.
This behaviour can't be achieved by the original Kim's model, because, even though the filters size can be changed, they will capture only features from adjacent words.

Regarding the architectural references in \cite{kim2014convolutional}, the filter's number $|f|$ and their dimension $(k,d)$, where $k$ is the kernel size and $d$ the dilation's unit, was optimized leading to the following results: $|f| = 5, f_1 = (2\times2,0), f_2 = (2\times2,3), f_3 = (3\times3,1), f_4 = (5\times5,1), f_5 = (7\times7,1)$.

\begin{figure}[h]
\footnotesize
\centering
\includegraphics[width=.75\columnwidth]{kim_cnn}
\caption{\cite{zhang2015sensitivity} Illustration of a Convolutional Neural Network (CNN) architecture for sentence classification}
\label{fig:kim}
\end{figure}

\subsubsection{Recurrent neural network.}
Long Short Term Memory (LSTM) and Bidirectional LSTM are types of Recurrent Neural Network (RNN) aiming at capture dynamic temporal behaviour.
This behaviour suggest us to use them for the stance detection, in particular we use straightforward architectures made of an embedded input layer followed by an LSTM layer of 128 units, terminated by a dense layer for both normal and bidirectional models.		% Methods description
\section{Evaluation} \label{sec:evaluation}

In this section we are going to illustrate the evaluation of developed systems regarding the modules design reported in \cref{sec:system}.
First we illustrate the metric proposed by organizers for system's evaluation (\Cref{subsec:metric}), then we outline empirical results produced by a 10-fold cross validation over the given data set (\Cref{subsec:tuning}), finally we report our performance at the shared task (\Cref{subsec:results}).

\subsection{Metrics} \label{subsec:metric}

System evaluation metrics were given by the organizers and reported here in the following \cref{eq:gender,eq:stance,eq:f1macro,eq:f1,eq:precision,eq:recall}. Their choice was to use an $F_{1-macro}$ measure for stance detection, due to class unbalance, while a categorical accuracy for the gender detection.

\begin{multicols}{2}
\begin{equation}  \label{eq:gender}
Gender = accuracy = \frac{\sum TP + \sum TN}{\sum sample}
\end{equation}

\begin{equation}  \label{eq:stance}
Stance = \frac{F_{1-macro}(Favor) + F_{1-macro}(Against)}{2}
\end{equation}

\begin{equation}  \label{eq:f1macro}
F_{1-macro}(L) = \frac{1}{|L|} \displaystyle\sum_{l\in L} F_1(y_l, \hat{y}_l)
\end{equation}	
	
\begin{equation} \label{eq:f1}
F_1 = 2 \cdot \frac{precision \cdot recall }{precision + recall}
\end{equation}

\begin{equation} \label{eq:precision}
precision = \frac{1}{|L|} \displaystyle\sum_{l\in L} Pr(y_l, \hat{y}_l)
\end{equation}

\begin{equation} \label{eq:recall}
recall = \frac{1}{|L|} \displaystyle\sum_{l\in L} R(y_l, \hat{y}_l)
\end{equation}
\end{multicols}

\noindent where $L$ is the set of classes, $y_l$ is the set of correct label and $\hat{y}_l$ is the set of predicted labels.


\subsection{Fine tuning process} \label{subsec:tuning}

Following, we describe the fine tuning process over possible combinations of pre-processing (\Cref{tab:preprocessing}), then we compare Kim's model against our extension (\Cref{tab:dilation}) and finally report the improvement over the use of a \emph{data augmentation} technique (\Cref{tab:augmentation}).
For brevity of information only the evaluation of Kim's model over Spanish stance detection is reported, in details, the results are calculated from averaging three runs of a 10-fold cross validation over the complete data set.
Nevertheless, the results obtained after the fine tuning process for all the models are reported in \cref{subsec:results}, where their development performances are compared against the ones obtained in the \emph{StanceCat} task.

Notations used in \Cref{tab:preprocessing} refer to the one introduced in \Cref{subsec:preprocessing}, where the listing of a notation means its use for the reported result. 
Regarding the tweet specific pre-processing, all the items have been substituted, with the exception for URL and RW that have been removed. We report the contribution of each analysed pre-processing alone.

\textbf{Luca to fill the table preprocessing with correct results}

\begin{table}[h]
\footnotesize
\caption{Pre-processing fine tuning for the Kim's model from a three run of 10-fold cross validation over the development set. Results are in terms of average $F_{1-macro}$ score. The processing technique that brought a model's improvement has its result in bold.}
\label{tab:preprocessing}
\centering
\begin{tabular}{l|cccccccccc}
\toprule
\hline
\multirow{2}{*}{Models}		& \multicolumn{10}{ c }{Pre-processing}       \\ 
		& Nothing	& ST	& SW	& URL	& RW	& MT	& HT	& NUM	& EM	& SM	\\
\hline
Kim		& 0.543		& 0.528	& \textbf{0.557}	&  \textbf{0.571}	& 0.533	&  \textbf{0.558}	& 0.540	&  \textbf{0.554}	& 0.537	& 0.539	\\
\hline
\bottomrule
\end{tabular}
\end{table}

\textbf{Luca qui tocca a te mettere prima i dati nella tabella e poi mettere gli insight in maniera di concetto e poi io ci do una forma da paper}

OLD from COSET
From the analysis of \Cref{tab:preprocessing} no absolute conclusion can be drawn, meaning that it wasn't possible to find a combination of pre-processing that gives the best performance for all the model, meaning that each model is highly sensible to the performed combination. Nevertheless, some relative observation can be made:
\begin{itemize}
	\item SW (i.e., removing spanish stop words) and NUM (i.e., substitute numbers with a constant string) leads to performance improvement to all the model respect to no pre-processing at all,
	\item ST (i.e., stemming) and HT (i.e., substitute hashtags with a constant string) decrease the performance of both models respect to no-preprocessing at all,
\end{itemize}

NEW for StanceCat
\begin{itemize}
	\item insight
\end{itemize}



\begin{table}[h]
\footnotesize
\caption{Comparison of Kim's and Dilated Kim respect their best pre-processing tuning for stance\&gender detection task. Results are averaged after three run of 10-fold cross validation over the development set in terms of averaged $F_{1-macro}$ score.}
\label{tab:dilation}
\centering
\begin{tabular}{l|cc|cc}
\toprule
\hline
\multirow{2}{*}{Models}		& \multicolumn{2}{c}{Stance}	& \multicolumn{2}{c}{Gender}\\
\cline{2-5}
							& ES		& CA		& ES		& CA		\\
\hline
Kim							& $0.625 (\pm0.019)$ & $0.602 (\pm0.019)$ & $0.625 (\pm0.019)$ & $0.602 (\pm0.019)$	\\
Dilated Kim					& $0.675 (\pm0.049)$ & $0.635 (\pm0.049)$ & $0.675 (\pm0.049)$ & $0.635 (\pm0.049)$	\\
\hline
\bottomrule
\end{tabular}
\end{table}

\textbf{As before, Luca qui tocca a te mettere prima i dati nella tabella (devi trovare tu la coerenza rispetto al fatto della data augmentation) e poi mettere gli insight in maniera di concetto e poi io ci do una forma da paper}

NEW for StanceCat
\begin{itemize}
	\item insight
\end{itemize}


Due to the fact that our development data set has a small number of samples, to train our models we decided to apply a \emph{data augmentation} technique that didn't rely over external data rather exploit the word embedding text representation. In details, we used a mixture of gaussian noise and batch normalization, first we \textbf{Luca put the description of where is putted the gn and how the batch normalization}. Results of this technique respect the Dilated Kim's model are reported in \cref{tab:augmentation}.

\begin{table}[h]
\footnotesize
\caption{Data augmentation study for Dilated Kim's model over the Spanish stance detection development dataset. Results are averaged after three run of 10-fold cross validation over the development set in terms of averaged $F_{1-macro}$ score. }
\label{tab:augmentation}
\centering
\begin{tabular}{c|ccc}
\toprule
\hline
System		& Nothing	& Gaussian noise	& Batch normalization	\\
\hline
Dilated Kim	& 0.556 ($\pm$ 0.012) & 0.556 ($\pm$ 0.012)	& 0.556 ($\pm$ 0.012)	\\
\hline
\bottomrule
\end{tabular}
\end{table}


\subsection{Competition results} \label{subsec:results}

For the system's submission, participants where allowed to send more than a model till a maximum of 5 possible runs, therefore in \cref{tab:stance,tab:gender} we report our best performing systems (tuned following the process in \cref{subsec:tuning}) for the StanceCat shared task.

Unfortunately, due to a submission error caught only after the official result were published, we didn't manage to be properly evaluated (the minus simbol in \cref{tab:stance,tab:gender}), therefore after the closing we asked organizers to evaluate some of our model to see how they would had performed (the test columns).

\begin{table}[h]
\footnotesize
\caption{Comparison of the best tuning model for the stance detection respect development and test set. The reported ranking refers to the absolute position over all submissions.}
\label{tab:stance}
\centering
\begin{tabular}{c|cc|cc|cc}
\toprule
\hline
\multirow{3}{*}{System}	& \multicolumn{2}{c|}{Development} & \multicolumn{4}{c}{Test}	\\
\cline{2-7}
						& \multirow{2}{*}{ES}	& \multirow{2}{*}{CA}	& \multicolumn{2}{c|}{ES} & \multicolumn{2}{c}{CA}	\\
\cline{4-7}
						&		&		& Score & Ranking & Score & Ranking \\
\hline
LSTM					& $0.443 (\pm0.012)$ & $0.489 (\pm0.012)$ & - & - & - & - \\
Bi-LSTM					& $0.564 (\pm0.035)$ & $0.566 (\pm0.035)$ & 0.410 & 17/31 & 0.386 & 20/31 \\
CNN						& $0.539 (\pm0.030)$ & $0.566 (\pm0.030)$ & - & - & - & - \\
Kim						& $0.625 (\pm0.019)$ & $0.602 (\pm0.019)$ & - & - & - & - \\
Dilated Kim				& $0.675 (\pm0.049)$ & $0.635 (\pm0.049)$ & - & - & - & - \\
\hline
\bottomrule
\end{tabular}
\end{table}

\textbf{Luca put some comments over the result in relation to the dilated kim model also to updates the data in the gender and to change the variance in stance detection.}


\begin{table}[h]
	\footnotesize
	\caption{Comparison of the best tuning model for the gender detection respect development and test set. The reported ranking refers to the absolute position over all submissions.}
	\label{tab:gender}
	\centering
	\begin{tabular}{c|cc|cc|cc}
		\toprule
		\hline
		\multirow{3}{*}{System}	& \multicolumn{2}{c|}{Development} & \multicolumn{4}{c}{Test}	\\
		\cline{2-7}
		& \multirow{2}{*}{ES}	& \multirow{2}{*}{CA}	& \multicolumn{2}{c|}{ES} & \multicolumn{2}{c}{CA}	\\
		\cline{4-7}
		&		&		& Score & Ranking & Score & Ranking \\
		\hline
		LSTM					& $0.443 (\pm0.012)$ & $0.489 (\pm0.012)$ & - & - & - & - \\
		Bi-LSTM					& $0.564 (\pm0.035)$ & $0.566 (\pm0.035)$ & 0.410 & 17/31 & 0.386 & 20/31 \\
		CNN						& $0.539 (\pm0.030)$ & $0.566 (\pm0.030)$ & - & - & - & - \\
		Kim						& $0.625 (\pm0.019)$ & $0.602 (\pm0.019)$ & - & - & - & - \\
		Dilated Kim				& $0.675 (\pm0.049)$ & $0.635 (\pm0.049)$ & - & - & - & - \\
		\hline
		\bottomrule
	\end{tabular}
\end{table}		% Methods evaluations
\section{Conclusions} \label{sec:conclusion}

In this paper we have presented our participation in the IberEval2017 Classification Of Spanish Election Tweets (COSET) shared task. Five different neural models were explored, in combination with 11 types of preprocessing. No preprocessing emerged to be the best with every kind of model, indicating that the preprocessing pipeline optimization has a big impact on results.
We also explored static vs non-static word embeddings and non-static vectors initialized with pre-trained vectors on a bigger corpus is the best performing combination.


	% Conclusions



% The bibliography is printed with \bibliography{}. With the command \bibliographystyle{} a style is picked.
%\bibliographystyle{plain} 		% for [1] cite style
%\bibliographystyle{plainnat}	% for [Surname et al] cite style
\bibliographystyle{splncs}		% FOR LNCS
%\bibliography{refs/references}
\begin{thebibliography}{1}

\bibitem{kim2014convolutional}
Kim, Yoon. "Convolutional neural networks for sentence classification." arXiv preprint arXiv:1408.5882 (2014).

\bibitem{joulin2016bag}
Joulin, Armand, et al. "Bag of tricks for efficient text classification." arXiv preprint arXiv:1607.01759 (2016).

\bibitem{nltk}
Edward Loper and Steven Bird. 2002. NLTK: the Natural Language Toolkit. In Proceedings of the ACL-02 Workshop on Effective tools and methodologies for teaching natural language processing and computational linguistics - Volume 1 (ETMTNLP '02), Vol. 1. Association for Computational Linguistics, Stroudsburg, PA, USA, 63-70. DOI=http://dx.doi.org/10.3115/1118108.1118117

%\bibitem{tweets-preprocessor}
%Preprocessor is a preprocessing library for tweet data written in Python, https://github.com/s/preprocessor

\bibitem{bojanowski2016enriching}
Bojanowski, Piotr and Grave, Edouard and Joulin, Armand and Mikolov, Tomas. "Enriching Word Vectors with Subword Information" arXiv preprint arXiv:1607.04606 (2016).

\bibitem{harris1954distributional}
Harris, Zellig S. "Distributional structure." Word 10.2-3 (1954): 146-162.

\bibitem{bazzanella2011oscillazioni}
Bazzanella, Carla. "Oscillazioni di informalità e formalità: scritto, parlato e rete." Formale e informale. La variazione di registro nella comunicazione elettronica. Roma: Carocci (2011): 68-83.

\bibitem{cerruti2013netspeak}
Cerruti, Massimo, and Cristina Onesti. "Netspeak: a language variety? Some remarks from an Italian sociolinguistic perspective." Languages go web: Standard and non-standard languages on the Internet (2013): 23-39.

% To better understand how it works go to 
% http://www.wildml.com/2015/11/understanding-convolutional-neural-networks-for-nlp/
\bibitem{zhang2015sensitivity}
Zhang, Ye, and Byron Wallace. "A sensitivity analysis of (and practitioners' guide to) convolutional neural networks for sentence classification." arXiv preprint arXiv:1510.03820 (2015).

\bibitem{bosco2016}
C. Bosco, M. Lai, V. Patti, F. Rangel, P. Rosso (2016) Tweeting in the Debate about Catalan Elections. In: Proc. LREC workshop on Emotion and Sentiment Analysis Workshop (ESA), LREC-2016, Portorož, Slovenia, May 23-28, pp. 67-70.

\bibitem{rangel2016overview}
F. Rangel, P. Rosso, B. Verhoeven, W. Daelemans, M. Potthast, B. Stein (2016) Overview of the 4th Author Profiling Task at PAN 2016: Cross-Genre Evaluations. In: Balog K., Cappellato L., Ferro N., Macdonald C. (Eds.) CLEF 2016 Labs and Workshops, Notebook Papers. CEUR Workshop Proceedings. CEUR-WS.org, vol. 1609, pp. 750-784.

\bibitem{mohammad2016stance}
Mohammad, Saif M., Parinaz Sobhani, and Svetlana Kiritchenko. "Stance and sentiment in tweets." arXiv preprint arXiv:1605.01655 (2016).

\bibitem{mohammad2016semeval}
Mohammad, Saif M., et al. "Semeval-2016 task 6: Detecting stance in tweets." Proceedings of SemEval 16 (2016).

\bibitem{yu2015multi}
Fisher Yu, Vladlen Koltun , “Multi-Scale Context Aggregation by Dilated Convolutions”

%Taulé M., Martí M.A., Rangel F., Rosso P., Bosco C., Patti V. 
%Overview of the task of Stance and Gender Detection in Tweets on Catalan Independence at IBEREVAL 2017. 
%In: Notebook Papers of 2nd SEPLN Workshop on Evaluation of Human Language Technologies for Iberian Languages (IBEREVAL), Murcia, Spain, September 19, CEUR Workshop Proceedings. CEUR-WS.org, 2017

\end{thebibliography}	% Conclusions

% To close your document, add the \end{document} command. Everything after this command will not be processed.
\end{document}