% This is the master file of the folder structure.

% First, the preamble needs to be called. This contains all the 'under the hood' stuff for your document.
% Use \input rather than than \include for .tex files, because \input can be nested and don't include a page break.
% This file contains your LaTeX preamble. A preamble is a part of your document where all required packages and macros can be defined. This needs to be done before the \begin{document} command.

% Documentclass:
% Standard LaTeX classes are: article, book, report, slides, and letter. These cover the basis, but are not best. More advanced users might want to try out the KOMA classes or the memoir class. Optional arguments: 10pt. The font size of the main content is set to 10pt with the option between [].
\documentclass{llncs}
\usepackage[utf8]{inputenc}

% Geometry:
% The papersize of the document is defined with the geometry package. Here, the size is set to A4 with a4paper. Other possibilities are a5paper, b5paper, letterpaper, legalpaper and executivepaper.
\usepackage[a4paper]{geometry}

% AMS math packages:
% Required for proper math display.
%\usepackage{amsmath,amsfonts,amsthm}	% conflict with the llncs package
\usepackage{amsmath,amsfonts}			% to avoid conflict with the llncs package
%\usepackage{amsfont} create a conflict so it's not used
\usepackage{amssymb}
\usepackage{mathtools}
% For case equation with nested label
% \usepackage{cases} NOT USEFUL



% Graphicx:
% If you want to include graphics in your document, the graphicx package is required.
\usepackage{graphicx}

% Graphic path declaration:
\graphicspath{{figs/}}

% Tabu:
% To test if this package could be the latest package for my work
\usepackage{tabu}

% [OLD] Booktabs:
% The booktabs package is needed for better looking tables. 
\usepackage{booktabs}

% [OLD] Tables:
\usepackage[table]{xcolor}
\usepackage{multirow}
%\usepackage{tabularx}

\newcommand{\win}{\cellcolor[gray]{0.9}}

% Enumeration:
% To manipulate the listing of object
\usepackage[inline]{enumitem}
\setlist*[enumerate,1]{%
	label=(\roman*),
}
% [Alternative]
%\usepackage{easylist}

% SIunitx:
% The SIunitx package enables the \SI{}{} command. It provides an easy way of working with (SI) units.
\usepackage{siunitx}

% URL:
% Clickable URL's can be made with this package: \url{}.
%\usepackage{url}

% Caption:
% For better looking captions. See caption documentation on how to change the format of the captions.
%\usepackage{caption}

% Hyperref:
% This package makes all references within your document clickable. By default, these references will become boxed and colored. This is turned back to normal with the \hypersetup command below.
\usepackage{hyperref}
\hypersetup{colorlinks=false,pdfborder=0 0 0}

% Cleveref:
% This package automatically detects the type of reference (equation, table, etc.) when the \cref{} command is used. It then adds a word in front of the reference, i.e. Fig. in front of a reference to a figure. With the \crefname{}{}{} command, these words may be changed.
\usepackage{cleveref}
\crefname{equation}{equation}{equations}
\crefname{figure}{figure}{figures}	
\crefname{table}{table}{tables}

% Pseudo code packages:
\usepackage[ruled,linesnumbered]{algorithm2e}
\usepackage{algorithmicx}
\usepackage{algpseudocode}

% Verbatim:
% Typewriter formatting and comment environment
\usepackage{verbatim}

% Bibliography style
%\usepackage{natbib}



% Command

% For the level structure depending of the 

% The title page is created with the command \maketitle which needs to be placed after the \begin{document} command. To create the titlepage, some entries are needed: the name of the autor is defined by \author{}, the title by the entry \title{} and the date by the command \date{}. Note that the current date is displayed with \today.
\author{Luca Ambrosini \and Giancarlo Nicol\`{o}}
\title{ALC}
\date{}

% All the actual content of your document should be placed after \begin{document} and before \end{document}. This content should be placed in the docs folder and can then be called with \input{docs/filename}.
\begin{document}

% Here the actual title page is printed, based on the given entries \author{}, \title{} and \date{}.
\maketitle

% The table of contents can be automatically generated with the \tableofcontents command. Note that you need to compile the document twice in order to see the changes in the table of contents.
\tableofcontents

% List of figures
\listoffigures
 
% List of tables
\listoftables

% The list of algorithm
\listofalgorithms

% The \input{} command reads and processes the indicated example.tex file. Note that docs/ locates the folder where the .tex file is stored.
\abstract

Version one

This paper describes an explorative approach to create from scratch a set of baselines for tackling tweet classification problems using natural language processing and machine learning techniques.
Our approach focuses on neural network models taken from state of the art that exploit different corpus preprocessing, tweets representations and features extraction methods.
Finally, we will discuss how we applied this methodology to the \emph{Classification Of Spanish Election Tweets (COSET)} task at IberEval 2017 and present the results we obtained

Version two

This paper describes our participation in the \emph{Classification Of Spanish Election Tweets (COSET)} task at IberEval 2017.
During the searching process for the best classification system to send for the competition, we developed a comparative study over possible combination of corpus preprocessing, text representations and classification models. After an initial models exploration, we focus our attention over specific neural models.
Interesting insight can be drawn from the comparative study helping future practitioner tackling tweets classification problems to create system baseline for their work.

Possible add for the report in general and 

handcrafted features

explain that we haven't put a proper module for handcrafted features, but our modular approach to solve the classification problem lead to an easy customization and integration of a proper module for these handcrafted features.		% Abstract
\section{Introduction} \label{sec:introduction}

Intro over nlp and text classification: motivation why this field is important and the actual challenge and open problems (will be top just to list some problem and their way to be solved, for example citing the different workshop that we had during the course)


Text classification is an important task in Natural Language Processing with many applications, such as web search, information retrieval, ranking and document classification (Deerwester et al., 1990; Pang and Lee, 2008). Recently, models based on neural networks have become increasingly popular (Kim, 2014; Zhang and LeCun, 2015; Conneau et al., 2016). While these models achieve very good performance in practice, they tend to be relatively slow both at train and test time, limiting their use on very large datasets.

The 

vedi citazioni su word embedding di mark e bag of word da maite

Introduce informally our task and list some way of tackling this problem from the different perspectives, maybe say that we de-construct tthe actual approach and cathegorized their inner process in: representation, preprocessing, model and post processing (that we haven't done).

Finally explain how we structured this report

In the following we firstly describe the 

In this paper we describe our participation for addressing
the PR-SOCO task. The rest of the paper is organized as
follows. Next section is devoted to dene the Personality
Recognition task. In Section 4 the model proposed is described.
Following, in Section 5, the results achieved are
presented. Finally, in Section 6 our results are discussed,
and future work is proposed.


	% Introduction
% A chapter named 'Your first document' is created
\chapter{Conclusion} \label{chapter:conclusion}

Here you have only the put all the concept that could be presented in the work and then in the literature review you will illustrate different proposed solution to the exposed topics.		% Introduction



% The bibliography is printed with \bibliography{}. With the command \bibliographystyle{} a style is picked.
\bibliographystyle{plain}
\bibliography{refs/references}
%\bibliography{refs/references_template} % Template

% To close your document, add the \end{document} command. Everything after this command will not be processed.
\end{document}