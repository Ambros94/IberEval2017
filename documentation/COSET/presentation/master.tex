% This is the master file of the folder structure.

% First, the preamble needs to be called. This contains all the 'under the hood' stuff for your document.
% Use \input rather than than \include for .tex files, because \input can be nested and don't include a page break.
% This file contains your LaTeX preamble. A preamble is a part of your document where all required packages and macros can be defined. This needs to be done before the \begin{document} command.

% Documentclass:
% Standard LaTeX classes are: article, book, report, slides, and letter. These cover the basis, but are not best. More advanced users might want to try out the KOMA classes or the memoir class. Optional arguments: 10pt. The font size of the main content is set to 10pt with the option between [].
%\documentclass[xcolor=table]{beamer}
\documentclass[xcolor=table,handout]{beamer}

\mode<presentation> {
	
	% The Beamer class comes with a number of default slide themes
	% which change the colors and layouts of slides. Below this is a list
	% of all the themes, uncomment each in turn to see what they look like.
	
	%\usetheme{default}
	%\usetheme{AnnArbor}
	%\usetheme{Antibes}
	%\usetheme{Bergen}
	%\usetheme{Berkeley}
	%\usetheme{Berlin}
	%\usetheme{Boadilla}
	%\usetheme{CambridgeUS}
	%\usetheme{Copenhagen}
	%\usetheme{Darmstadt}
	%\usetheme{Dresden}
	%\usetheme{Frankfurt}
	%\usetheme{Goettingen}
	%\usetheme{Hannover}
	%\usetheme{Ilmenau}
	%\usetheme{JuanLesPins}
	%\usetheme{Luebeck}
	\usetheme{Madrid}
	%\usetheme{Malmoe}
	%\usetheme{Marburg}
	%\usetheme{Montpellier}
	%\usetheme{PaloAlto}
	%\usetheme{Pittsburgh}
	%\usetheme{Rochester}
	%\usetheme{Singapore}
	%\usetheme{Szeged}
	%\usetheme{Warsaw}
	
	% As well as themes, the Beamer class has a number of color themes
	% for any slide theme. Uncomment each of these in turn to see how it
	% changes the colors of your current slide theme.
	
	%\usecolortheme{albatross}
	%\usecolortheme{beaver}
	%\usecolortheme{beetle}
	%\usecolortheme{crane}
	%\usecolortheme{dolphin}
	%\usecolortheme{dove}
	%\usecolortheme{fly}
	%\usecolortheme{lily}
	%\usecolortheme{orchid}
	%\usecolortheme{rose}
	%\usecolortheme{seagull}
	%\usecolortheme{seahorse}
	%\usecolortheme{whale}
	%\usecolortheme{wolverine}
	
	%\setbeamertemplate{footline} % To remove the footer line in all slides uncomment this line
	%\setbeamertemplate{footline}[page number] % To replace the footer line in all slides with a simple slide count uncomment this line
	
	%\setbeamertemplate{navigation symbols}{} % To remove the navigation symbols from the bottom of all slides uncomment this line
}

\usepackage{graphicx} % Allows including images
\usepackage{booktabs} % Allows the use of \toprule, \midrule and \bottomrule in tables

% ADDED BY MYSELF
\graphicspath{{figs/}}
\usepackage{tikz} % for the logo management

%Pseudo code package
\usepackage[ruled,linesnumbered]{algorithm2e}
\usepackage{algorithmicx}
\usepackage{algpseudocode}

% Table
%\usepackage[table]{xcolor}
\usepackage{tabularx}
\usepackage{multirow}

% Diagram
\usepackage{tikz}
\usepackage{verbatim}
%\usepackage[margin=15mm]{geometry}
\usetikzlibrary{shapes,arrows,fit,calc,positioning}

% Algorithm
%\usepackage[ruled,dotocloa,algochapter]{algorithm2e}
%\usepackage[boxruled,vlined,linesnumbered]{algorithm2e}
%\usepackage[linesnumbered]{algorithmicx}
%\usepackage{algpseudocode}
\usepackage[]{algorithm2e}
%\resetcounteronoverlays{algocf}

% algorithm environment
\newenvironment{agreement}[1][htb]
{\renewcommand{\algorithmcfname}{Agreement}% Update algorithm name
	\begin{algorithm}[#1]%
	}{\end{algorithm}}

% References
\usepackage{cleveref}

% Font 
\usepackage{lmodern}	% textttt plus bold
\usepackage{bm}			% math plus bold with \bm

% Footnotes
%\usepackage[perpage]{footmisc}	% reset the footnote counter after each slide

% Slides command
\AtBeginSection[]
{
	\begin{frame}
	\frametitle{Table of Contents}
	\tableofcontents[currentsection]
\end{frame}
}

\AtBeginSubsection[]
{
\begin{frame}
\frametitle{Table of Contents}
\tableofcontents[currentsection,currentsubsection]
\end{frame}
}


% SCRIPT

% table color
\newcommand{\win}{\cellcolor[gray]{0.9}}


% Copyright
\newcommand{\cmark}{$^{\tiny{\textcopyright}}$}

% Registered trademark 
\newcommand{\rmark}{$^{\tiny{\textregistered}}$}

\newcommand{\bfttt}[1]{\textbf{\texttt{#1}}}

% The title page is created with the command \maketitle which needs to be placed after the \begin{document} command. To create the titlepage, some entries are needed: the name of the autor is defined by \author{}, the title by the entry \title{} and the date by the command \date{}. Note that the current date is displayed with \today.

\title[ACS Scheduling]{
Ant colony system for scheduling problem
} % The short title appears at the bottom of every slide, the full title is only on the title page

\author[Ambrosini \and Nicol\`{o}]{Ambrosini,~Luca \and Nicol\`{o},~Giancarlo %\and Miguel~A.~Salido \and Adriana~Giret \and Federico~Barber
} % Your name
\institute[UPV] % Your institution as it will appear on the bottom of every slide, may be shorthand to save space
{
	\inst{1}%
	Universitat Polit\`{e}cnica de Val\`{e}ncia % Your institution for the title page
	%\and
	%\inst{2}%
	%Instituto Universitario de Autom\'{a}tica e Inform\'{a}tica Industrial \\
	%\medskip
	%\textit{giani1@dsic.upv.es, msalido@dsic.upv.es, agiret@dsic.upv.es, fbarber@dsic.upv.es} % Your email address
}

% logo of my university
\titlegraphic{
	%\includegraphics[width=1cm]{AI2_logo.jpg}\hspace*{4.75cm}~%
	\includegraphics[width=2cm]{UPV_logo.png}%
}

%\logo{
%	\includegraphics[height=0.8cm]{ICAPS_logo.png}%
%}

%\date{\today} % Date, can be changed to a custom date
\date[May, 8th 2017]{May, 8th 2017 }




% All the actual content of your document should be placed after \begin{document} and before \end{document}. This content should be placed in the docs folder and can then be called with \input{docs/filename}.
\begin{document}

% Here the actual title page is printed, based on the given entries \author{}, \title{} and \date{}.
%\maketitle

% The table of contents can be automatically generated with the \tableofcontents command. Note that you need to compile the document twice in order to see the changes in the table of contents.
%\tableofcontents

% List of figures
%\listoffigures
 
% List of tables
%\listoftables

% The list of algorithm
%\listofalgorithms

% The \input{} command reads and processes the indicated example.tex file. Note that docs/ locates the folder where the .tex file is stored.
%----------------------------------------------------------------------------------------
%	INTRODUCTION SLIDES
%----------------------------------------------------------------------------------------

\begin{frame}
\titlepage % Print the title page as the first slide
\end{frame}


\begin{frame} \frametitle{Overview} % Table of contents slide, comment this block out to remove it
\tableofcontents[hideallsubsections] % Throughout your presentation, if you choose to use \section{} and \subsection{} commands, these will automatically be printed on this slide as an overview of your presentation
%\tableofcontents[currentsection]
\end{frame}

%------------------------------------------------
\section{Introduction}
%------------------------------------------------


\begin{frame} \frametitle{Idea}

\begin{block}{Aim}
	Novel work to tackle literature leaks in the scheduling domain applying meteheuristic approaches.
\end{block}

\medskip
\begin{columns}[c]
	\begin{column}[c]{6cm}
		Scheduling domain:
		\begin{itemize}
			\item Energy aware scheduling
			\item Realistic problem (setup, release, due date)
		\end{itemize}
	\end{column}
	\pause
	\begin{column}[c]{6cm}
		Metaheuristic:
		\begin{itemize}
			\item Ant Colony Optimization
			\item Ant Colony System
		\end{itemize}
	\end{column}
\end{columns}

\end{frame}


\begin{frame} \frametitle{Motivation}
\pause
Why scheduling problem?
\begin{itemize}[<+->]
	\item most of daily life problem can easily map in scheduling problem
	\begin{itemize}
		\item industrial/manufacturing environment 
		\item supply-chain management
		\item space orbit 
	\end{itemize}
	\item optimization of shared resources
	\item search for efficiency and cost reduction
	\item avoid critical situation
\end{itemize}

\begin{columns}[c]
	\begin{column}[c]{6cm}
		\begin{figure}
			\centering
			\includegraphics[width=.8\columnwidth]{boeing}
			%\caption{Interaction sequence to solve the distributed scheduling problem}
		\end{figure}
	\end{column}
	\pause
	\begin{column}[c]{6cm}
		\begin{figure}
			\centering
			\includegraphics[width=.8\columnwidth]{nasa}
			%\caption{Interaction sequence to solve the distributed scheduling problem}
		\end{figure}
	\end{column}
\end{columns}

\end{frame}			% Intoduction (title, table of contents)
%------------------------------------------------
\section{Problem}
%------------------------------------------------

\begin{frame}
\frametitle{Problem description}
The problem consists in \emph{scheduling a set of job}, on a set of \emph{unrelated parallel machines}, while optimizing objective functions.

\medskip

%Each job has a \emph{release date} (hard constraint), a \textbf{due date} (soft constraint), and a \emph{weighted penalty cost} depending on the generete delay after its due date.

Job characteristics are: 
\begin{itemize}
	\item \emph{release date} (hard constraint)
	\item \emph{due date} (soft constraint)
	\item \emph{weighted penalty cost}%, depending on the generated delay after due date.
\end{itemize}

\medskip

Each job have to be processed by only one machine, selected from a set of eligible ones, where \emph{processing time} and \emph{energy consumption} depend by the selected machine.

\medskip

In order to process a job the machines must be setup and this setup time depends on the \emph{machine} and the \emph{previously scheduled job} on it.

\begin{displaymath}
R_m | M_j,p_{ij},E_{ij},r_j,d_j,w_j,s_{ijk} | \displaystyle\sum w_j T_j, \displaystyle\sum E_{ij}, \displaystyle\sum S_{ijk}
\end{displaymath}

\end{frame}


\begin{frame} \frametitle{Problem example}
\begin{figure}[t]
	\label{fig:example_sched}
	\centering
	\includegraphics[width=.75\columnwidth]{example_sched}
	\caption{Scheduling example 1}
\end{figure}
\end{frame}

\begin{frame} \frametitle{Problem example}
\begin{figure}[t]
	\label{fig:example_sched2}
	\centering
	\includegraphics[width=.75\columnwidth]{example_sched2}
	\caption{Scheduling example 2}
\end{figure}
\end{frame}

\begin{frame} \frametitle{Problem objectives}
In relation to a possible scheduling solution $s$, three different objective functions can be stated:
\begin{itemize}
\item $TT(s)$: total weighted tardiness of the jobs,
\item $EN(s)$: energy consumption,
\item $ST(s)$: total setup time.
\end{itemize} 
The aim is to simultaneously optimized them in a multi-objective function:
\begin{equation}
\label{eq_1:multi_objective_function}
s^* = \operatorname*{arg\,min}_{s \in S} [TT(s),EN(s),ST(s)] 
\end{equation}
where $S$ denotes the feasibility space for the problem solutions.

\end{frame}


\begin{frame}
\frametitle{Problem objectives}

The three objective functions in (\ref{eq_1:multi_objective_function}) are aggregated into a scalar normalized objective function F:
\begin{equation}
\label{eq_2:normalized_multi_objective_function}
F(s) = \displaystyle\sum_{g=1}^{3} \Pi_{g} \cdot \frac{f_g(s) - f^{-}_{g}}{f^{+}_{g} - f^{-}_{g}}
\end{equation}
where:
\begin{itemize}
\item $f_g(s)$,~$g\in\{1,2,3\}$, represents $TT(s)$, $EN(s)$ and $ST(s)$.

\item $f^{-}_{g}$ represents the best (\emph{minimum}) value %for the $g$-th component when it is optimized individually
\item $f^{+}_{g}$ is an estimation of the worse value for $f_{g}(s)$%, that can be fixed as $f^{+}_{g} = \operatorname*{max}_{h\neq g} f_{g}(s_{h}^{*})$%, where $(s_{h}^{*})$ is the optimal solution found when the objective $f_{h}(s)$ is individually optimized
\item $\Pi_{g}$,~$g\in\{1,2,3\}$, represents the relative importance given by the decision maker to the different objective, where $\sum_{g}\Pi_{g} = 1$
\end{itemize}

\end{frame}		% Problem
%%------------------------------------------------
\section{MIP model}
%------------------------------------------------

\begin{frame} \frametitle{MIP model} 

\begin{itemize}[<+->]
\item use basic functionality of IBM ILOG CPLEX optimization studio
\item comprehende basic constructs of the OPL language
\item apply basic OPL constructs over a constraint satisfaction problem (CSP)
\item comprehende advanced constructs of the OPL language
\item recognize a job-shop scheduling problems
\item apply advanced constructs to model a job-shop scheduling problem
\end{itemize}

\end{frame}			% MIP model
%------------------------------------------------
\section{Ant colony system}
%------------------------------------------------

\begin{frame} \frametitle{Ant colony} 

\begin{itemize}[<+->]
	\item Meta-heuristic proposed by [Colorni, et Al], improved by [Dorigo, Gambardella, et al.]
	\item Mimicry of ants behavior to find optimal path between nest and food
	\begin{itemize}
		\item Transition rules
	\end{itemize}
	\item Indirect communication between ants through pheromones amount
	\begin{itemize}
		\item Update rules
	\end{itemize}
	\item State of the art in various optimization problem
\end{itemize}

\end{frame}


\begin{frame}\frametitle{Ant exaple}
\begin{figure}[t]
	\label{fig:ants}
	\centering
	\includegraphics[width=.75\columnwidth]{ants_example}
	\caption{How real ants find a shortest path [Dorigo, Gambardella, et al.].}
\end{figure}
\end{frame}


\begin{frame} \frametitle{Transition rules} 

\begin{block}{Ant Colony Optimization}
\begin{equation*}
P_{ij}^{k} = \frac{\tau_{ij}^{\alpha}~\eta_{ij}^{\beta}}{\sum_{l\in\Psi}\tau_{il}^{\alpha}~\eta_{il}^{\beta}}
\end{equation*}	
\end{block}

\begin{block}{Ant Colony System}
Pseudo random proportional rule:
\begin{equation*}
j = 
\begin{cases} 
\displaystyle\operatorname*{arg\,max}_{j\in J(i)} \{ {[\tau_{ij}]\cdot[\eta_{ij}]^{\beta}} \} & \text{if } q\leq q_0 \quad\quad \text{(exploitation)} \\
S & otherwise \quad \text{(biased exploration)}
\end{cases}
\end{equation*}
where $S$ is a random variable selected according to $P_{ij}^{k}$ 
\end{block}

\end{frame}



\begin{frame} \frametitle{Update rules} 

\begin{block}{Global update rule}
\begin{equation*}
\tau_{ij} \leftarrow (1 - \alpha)\cdot\tau_{ij}~+~\alpha\cdot\tau_{ij}^{k}
\end{equation*}

\begin{equation*}
\Delta\tau_{ij}^{k} = 
\begin{cases} 
1/L_{gb} & \text{if arc}~(i,j)\in~\text{global-best-tour} \\
0		 & \text{Otherwise}
\end{cases}
\end{equation*}

\end{block}

\begin{block}{Local update rule}

\begin{equation*}
\tau_{ij} \leftarrow (1 - \rho)\cdot\tau_{ij}~+~\rho\cdot\tau_{ij}^{k}
\end{equation*}

\begin{equation*}
\Delta\tau_{ij}^{k} = 
\begin{cases} 
1/L_{init} 	& \text{if arc}~(i,j)\in~\text{initial-tour} \\
0    		& \text{Otherwise}
\end{cases}
\end{equation*}
\end{block}

\end{frame}


\begin{frame} \frametitle{Solution encoding} 

\begin{block}{Machine assignment}
\begin{equation} \label{eq:stage1} \tag{stage 1}
S_1 =[~3~2~3~1~4~3~4~2~1~4]
\end{equation}
(interpretation) machine three ($m_3$) has assigned jobs 1, 3 and 6
\end{block}

\begin{block}{Jobs sequencing}
\begin{equation}\label{eq:stage2} \tag{stage 2}
S_2 =
\begin{bmatrix}
9  &  4  &  0 &  0 &  0 &  0 &  0 &  0 &  0 &  0 \\
8  &  2  &  0 &  0 &  0 &  0 &  0 &  0 &  0 &  0 \\
6  &  3  &  1 &  0 &  0 &  0 &  0 &  0 &  0 &  0 \\
5  & 10  &  7 &  0 &  0 &  0 &  0 &  0 &  0 &  0 
\end{bmatrix}
\end{equation}	
(interpretation) machine one ($m_1$) will process jobs in the following order: $9 \rightarrow 4$
\end{block}

\end{frame}


\begin{frame} \frametitle{Transition definition (Stage 1)} 

\begin{block} {Transition rule}
\begin{equation}\label{eq:acsTauStage1} 
j = 
\begin{cases} 
\displaystyle\operatorname*{arg\,max}_{j\in J(i)} \{ {[\tau_{jk}^{I}]\cdot[\eta_{jk}^{I}]^{\beta}} \} & \text{if } q\leq q_0 \quad\quad \text{(exploitation)} \\
S & otherwise \quad \text{(biased exploration)}
\end{cases}
\end{equation}
\end{block}

\begin{block}{Visibility metric}
\begin{columns}[c]
	\begin{column}[c]{6cm}
	\begin{equation} \label{eq:acsEta1Stage1}
	\eta_{jk}^{I} = \frac{1}{P_{jk}}
	\end{equation}	
	\end{column}
	\pause
	\begin{column}[c]{6cm}
	\begin{equation} \label{eq:acsEta2Stage1}
	\eta_{jk}^{I} = \frac{1}{
		\left[ \frac{P_{jk}}{\operatorname*{max}_{m\in M_{j}}(P_{jm})} +
		\frac{E_{jk}}{\operatorname*{max}_{m\in M_{j}}(E_{jm})}\right]}
	\end{equation}
	[Tonelli, Salido, et al. 2016]
	\end{column}
\end{columns}
\end{block}

\end{frame}


\begin{frame} \frametitle{Transition definition (Stage 2)} 

\begin{block} {Transition rule}
\begin{equation} \label{eq:acsTauStage2}
j = 
\begin{cases} 
\displaystyle\operatorname*{arg\,max}_{j\in J(i)} \{ {[\tau_{ij}^{II,k}]\cdot[\eta_{ij}^{II,k}]^{\beta}} \} & \text{if } q\leq q_0 \quad\quad \text{(exploitation)} \\
S & otherwise \quad \text{(biased exploration)}
\end{cases}
\end{equation}
\end{block}

\begin{block}{Visibility metric}
	\begin{columns}[c]
		\begin{column}[c]{6cm}
		\begin{equation} \label{eq:acsEta1Stage2}
		\eta_{ij}^{II,k} = \frac{1}{s_{ijk}}
		\end{equation}
		\end{column}
		\pause
		\begin{column}[c]{6cm}
		\begin{equation} \label{eq:acsEta2Stage2}
		\eta_{ij}^{II,k} = \frac{1}{
			\left[ \frac{s_{ijk}}{\operatorname*{max}_{j'\in J_{k}}(s_{ij'k})} +
			\frac{r_{j}}{\operatorname*{max}_{j'\in J_{k}}(r_{j'})}\right]}
		\end{equation}
		\end{column}
	\end{columns}
\end{block}

\end{frame}


\begin{frame} \frametitle{Pheromone global update} 

\begin{block} {Stage 1}
\begin{equation} \label{eq:acsTauStage1UpdateGlobal}
\tau_{jk}^{I} \leftarrow (1 - \alpha)\cdot\tau_{jk}^{I} +\alpha\cdot\Delta\tau_{jk}^{I,Best}
\end{equation}

\begin{equation}
\Delta\tau_{jk}^{I,Best} = 
\begin{cases} 
1/F(s_{gb}) & \text{if arc}~(j,k)\in~\text{global-best-schedule} \\
0			& \text{Otherwise}
\end{cases}
\end{equation}

\end{block}

\begin{block}{Stage 2}

\begin{equation} \label{eq:acsTauStage2UpdateGlobal}
\tau_{ij}^{II,k} \leftarrow (1 - \alpha)\cdot\tau_{ij}^{II,k} +\alpha\cdot\Delta\tau_{ij}^{II,Best}
\end{equation}
\begin{equation}
\Delta\tau_{ij}^{II,Best} = 
\begin{cases} 
1/F(s_{gb}) & \text{if arc}~(i,j,k)\in~\text{global-best-schedule} \\
0			& \text{Otherwise}
\end{cases}
\end{equation}
\end{block}


\end{frame}


\begin{frame} \frametitle{Pheromone local update} 

\begin{block} {Stage 1}
\begin{equation} \label{eq:acsTauStage1UpdateLocal}
\tau_{jk}^{I} \leftarrow (1 - \rho)\cdot\tau_{jk}^{I} +\rho\cdot\Delta\tau_{jk}^{I,Best}
\end{equation}

	
\begin{equation}
\Delta\tau_{jk}^{I,Best} = 
\begin{cases} 
1/F(s_{init}) & \text{if arc}~(j,k)\in~\text{initial-best-schedule} \\
0			& \text{Otherwise}
\end{cases}
\end{equation}

	
\end{block}

\begin{block}{Stage 2}
	
\begin{equation} \label{eq:acsTauStage2UpdateLocal}
\tau_{ij}^{II,k} \leftarrow (1 - \rho)\cdot\tau_{ij}^{II,k} +\rho\cdot\Delta\tau_{ij}^{II,Best}
\end{equation}

\begin{equation}
\Delta\tau_{ij}^{II,Best} = 
\begin{cases} 
1/F(s_{init}) & \text{if arc}~(i,j,k)\in~\text{initial-best-schedule} \\
0			& \text{Otherwise}
\end{cases}
\end{equation}


\end{block}


\end{frame}


%\subsection{Local Search}

\begin{frame}[fragile] \frametitle{Local search procedure} 


\begin{algorithm}[H]
	\footnotesize
	\caption{Local search procedure}
	\label{alg:acoLocalSearch}
	Set $LocalIteration = 1$\;
	\While{$LocalIteration \leq MaxLocalterations$}{
		Generate random variable ($rv$) from $U(0,1)$\;
		\eIf{$rv < 0.5$}{
			generate neighboring solution for $S_1(Ant)$: $N_1(Ant) \wedge x = 1$\;
		}{
			generate neighboring solution for $S_2(Ant)$: $N_2(Ant) \wedge x = 2$
		}
		Determine $F(N_x(Ant))$\;
		\If{$F(N_x(Ant)) \leq F(Ant)$}{
			$S_x \leftarrow N_x(Ant)$\;
		}
		$LocalIteration = LocalIteration + 1$
	}
\end{algorithm}

\end{frame}


%\subsection{ACS workflow}

\begin{frame}[fragile] \frametitle{Ant colony system workflow} 

\begin{algorithm}[H]
	\footnotesize
	\caption{ACS workflow}
	\label{alg:acoWorkflow}
	Populate the paths with specified pheromone amounts $(\tau_{jk}^{I},\tau_{ij}^{II,k})$\;
	\While{$not~StopCriteria$}{
		Set $Step = 1$\;
		\While{$Step \leq MaxSteps$}{
			\For{$Ant \in Ants$}{
				Solve $Step$ for Stage 1 (Assignment): find $S_1$ according to \cref{eq:acsTauStage1,eq:acsEta1Stage1,eq:acsEta2Stage1}\;
				Solve $Step$ for Stage 2 (Sequencing): find $S_2$ according to \cref{eq:acsTauStage2,eq:acsEta1Stage2,eq:acsEta2Stage2}\;		
			}	
			Update pheromone amounts locally according to \cref{eq:acsTauStage1UpdateLocal,eq:acsTauStage2UpdateLocal}\; 
		}
		Calculate $F(Ants)$ that are associated with $S_1$ and $S_2$\;		
		Execute local search procedure for all ants\;
		Update pheromone amounts globally according to \cref{eq:acsTauStage1UpdateGlobal,eq:acsTauStage2UpdateGlobal}\;
	}
\end{algorithm}


\end{frame}
			% Ant colony system
\section{Evaluation} \label{sec:evaluation}

In this section we are going to illustrate the evaluation of developed systems regarding the modules design reported in \cref{sec:system}.
First we illustrate the metric proposed by organizers for system's evaluation (\Cref{subsec:metric}), then we outline empirical results produced by a 10-fold cross validation over the given data set (\Cref{subsec:tuning}), finally we report our performance at the shared task (\Cref{subsec:results}).

\subsection{Metrics} \label{subsec:metric}

System evaluation metrics were given by the organizers and reported here in the following \cref{eq:gender,eq:stance,eq:f1macro,eq:f1,eq:precision,eq:recall}. Their choice was to use an $F_{1-macro}$ measure for stance detection, due to class unbalance, while a categorical accuracy for the gender detection.

\begin{multicols}{2}
\begin{equation}  \label{eq:gender}
Gender = accuracy = \frac{\sum TP + \sum TN}{\sum sample}
\end{equation}

\begin{equation}  \label{eq:stance}
Stance = \frac{F_{1-macro}(Favor) + F_{1-macro}(Against)}{2}
\end{equation}

\begin{equation}  \label{eq:f1macro}
F_{1-macro}(L) = \frac{1}{|L|} \displaystyle\sum_{l\in L} F_1(y_l, \hat{y}_l)
\end{equation}	
	
\begin{equation} \label{eq:f1}
F_1 = 2 \cdot \frac{precision \cdot recall }{precision + recall}
\end{equation}

\begin{equation} \label{eq:precision}
precision = \frac{1}{|L|} \displaystyle\sum_{l\in L} Pr(y_l, \hat{y}_l)
\end{equation}

\begin{equation} \label{eq:recall}
recall = \frac{1}{|L|} \displaystyle\sum_{l\in L} R(y_l, \hat{y}_l)
\end{equation}
\end{multicols}

\noindent where $L$ is the set of classes, $y_l$ is the set of correct label and $\hat{y}_l$ is the set of predicted labels.


\subsection{Fine tuning process} \label{subsec:tuning}

Following, we describe the fine tuning process over possible combinations of pre-processing (\Cref{tab:preprocessing}), then we compare Kim's model against our extension (\Cref{tab:dilation}) and finally report the improvement over the use of a \emph{data augmentation} technique (\Cref{tab:augmentation}).
For brevity of information only the evaluation of Kim's model over Spanish stance detection is reported, in details, the results are calculated from averaging three runs of a 10-fold cross validation over the complete data set.
Nevertheless, the results obtained after the fine tuning process for all the models are reported in \cref{subsec:results}, where their development performances are compared against the ones obtained in the \emph{StanceCat} task.

Notations used in \Cref{tab:preprocessing} refer to the one introduced in \Cref{subsec:preprocessing}, where the listing of a notation means its use for the reported result. 
Regarding the tweet specific pre-processing, all the items have been substituted, with the exception for URL and RW that have been removed. We report the contribution of each analysed pre-processing alone.

\textbf{Luca to fill the table preprocessing with correct results}

\begin{table}[h]
\footnotesize
\caption{Pre-processing fine tuning for the Kim's model from a three run of 10-fold cross validation over the development set. Results are in terms of average $F_{1-macro}$ score. The processing technique that brought a model's improvement has its result in bold.}
\label{tab:preprocessing}
\centering
\begin{tabular}{l|cccccccccc}
\toprule
\hline
\multirow{2}{*}{Models}		& \multicolumn{10}{ c }{Pre-processing}       \\ 
		& Nothing	& ST	& SW	& URL	& RW	& MT	& HT	& NUM	& EM	& SM	\\
\hline
Kim		& 0.543		& 0.528	& \textbf{0.557}	&  \textbf{0.571}	& 0.533	&  \textbf{0.558}	& 0.540	&  \textbf{0.554}	& 0.537	& 0.539	\\
\hline
\bottomrule
\end{tabular}
\end{table}

\textbf{Luca qui tocca a te mettere prima i dati nella tabella e poi mettere gli insight in maniera di concetto e poi io ci do una forma da paper}

OLD from COSET
From the analysis of \Cref{tab:preprocessing} no absolute conclusion can be drawn, meaning that it wasn't possible to find a combination of pre-processing that gives the best performance for all the model, meaning that each model is highly sensible to the performed combination. Nevertheless, some relative observation can be made:
\begin{itemize}
	\item SW (i.e., removing spanish stop words) and NUM (i.e., substitute numbers with a constant string) leads to performance improvement to all the model respect to no pre-processing at all,
	\item ST (i.e., stemming) and HT (i.e., substitute hashtags with a constant string) decrease the performance of both models respect to no-preprocessing at all,
\end{itemize}

NEW for StanceCat
\begin{itemize}
	\item insight
\end{itemize}



\begin{table}[h]
\footnotesize
\caption{Comparison of Kim's and Dilated Kim respect their best pre-processing tuning for stance\&gender detection task. Results are averaged after three run of 10-fold cross validation over the development set in terms of averaged $F_{1-macro}$ score.}
\label{tab:dilation}
\centering
\begin{tabular}{l|cc|cc}
\toprule
\hline
\multirow{2}{*}{Models}		& \multicolumn{2}{c}{Stance}	& \multicolumn{2}{c}{Gender}\\
\cline{2-5}
							& ES		& CA		& ES		& CA		\\
\hline
Kim							& $0.625 (\pm0.019)$ & $0.602 (\pm0.019)$ & $0.625 (\pm0.019)$ & $0.602 (\pm0.019)$	\\
Dilated Kim					& $0.675 (\pm0.049)$ & $0.635 (\pm0.049)$ & $0.675 (\pm0.049)$ & $0.635 (\pm0.049)$	\\
\hline
\bottomrule
\end{tabular}
\end{table}

\textbf{As before, Luca qui tocca a te mettere prima i dati nella tabella (devi trovare tu la coerenza rispetto al fatto della data augmentation) e poi mettere gli insight in maniera di concetto e poi io ci do una forma da paper}

NEW for StanceCat
\begin{itemize}
	\item insight
\end{itemize}


Due to the fact that our development data set has a small number of samples, to train our models we decided to apply a \emph{data augmentation} technique that didn't rely over external data rather exploit the word embedding text representation. In details, we used a mixture of gaussian noise and batch normalization, first we \textbf{Luca put the description of where is putted the gn and how the batch normalization}. Results of this technique respect the Dilated Kim's model are reported in \cref{tab:augmentation}.

\begin{table}[h]
\footnotesize
\caption{Data augmentation study for Dilated Kim's model over the Spanish stance detection development dataset. Results are averaged after three run of 10-fold cross validation over the development set in terms of averaged $F_{1-macro}$ score. }
\label{tab:augmentation}
\centering
\begin{tabular}{c|ccc}
\toprule
\hline
System		& Nothing	& Gaussian noise	& Batch normalization	\\
\hline
Dilated Kim	& 0.556 ($\pm$ 0.012) & 0.556 ($\pm$ 0.012)	& 0.556 ($\pm$ 0.012)	\\
\hline
\bottomrule
\end{tabular}
\end{table}


\subsection{Competition results} \label{subsec:results}

For the system's submission, participants where allowed to send more than a model till a maximum of 5 possible runs, therefore in \cref{tab:stance,tab:gender} we report our best performing systems (tuned following the process in \cref{subsec:tuning}) for the StanceCat shared task.

Unfortunately, due to a submission error caught only after the official result were published, we didn't manage to be properly evaluated (the minus simbol in \cref{tab:stance,tab:gender}), therefore after the closing we asked organizers to evaluate some of our model to see how they would had performed (the test columns).

\begin{table}[h]
\footnotesize
\caption{Comparison of the best tuning model for the stance detection respect development and test set. The reported ranking refers to the absolute position over all submissions.}
\label{tab:stance}
\centering
\begin{tabular}{c|cc|cc|cc}
\toprule
\hline
\multirow{3}{*}{System}	& \multicolumn{2}{c|}{Development} & \multicolumn{4}{c}{Test}	\\
\cline{2-7}
						& \multirow{2}{*}{ES}	& \multirow{2}{*}{CA}	& \multicolumn{2}{c|}{ES} & \multicolumn{2}{c}{CA}	\\
\cline{4-7}
						&		&		& Score & Ranking & Score & Ranking \\
\hline
LSTM					& $0.443 (\pm0.012)$ & $0.489 (\pm0.012)$ & - & - & - & - \\
Bi-LSTM					& $0.564 (\pm0.035)$ & $0.566 (\pm0.035)$ & 0.410 & 17/31 & 0.386 & 20/31 \\
CNN						& $0.539 (\pm0.030)$ & $0.566 (\pm0.030)$ & - & - & - & - \\
Kim						& $0.625 (\pm0.019)$ & $0.602 (\pm0.019)$ & - & - & - & - \\
Dilated Kim				& $0.675 (\pm0.049)$ & $0.635 (\pm0.049)$ & - & - & - & - \\
\hline
\bottomrule
\end{tabular}
\end{table}

\textbf{Luca put some comments over the result in relation to the dilated kim model also to updates the data in the gender and to change the variance in stance detection.}


\begin{table}[h]
	\footnotesize
	\caption{Comparison of the best tuning model for the gender detection respect development and test set. The reported ranking refers to the absolute position over all submissions.}
	\label{tab:gender}
	\centering
	\begin{tabular}{c|cc|cc|cc}
		\toprule
		\hline
		\multirow{3}{*}{System}	& \multicolumn{2}{c|}{Development} & \multicolumn{4}{c}{Test}	\\
		\cline{2-7}
		& \multirow{2}{*}{ES}	& \multirow{2}{*}{CA}	& \multicolumn{2}{c|}{ES} & \multicolumn{2}{c}{CA}	\\
		\cline{4-7}
		&		&		& Score & Ranking & Score & Ranking \\
		\hline
		LSTM					& $0.443 (\pm0.012)$ & $0.489 (\pm0.012)$ & - & - & - & - \\
		Bi-LSTM					& $0.564 (\pm0.035)$ & $0.566 (\pm0.035)$ & 0.410 & 17/31 & 0.386 & 20/31 \\
		CNN						& $0.539 (\pm0.030)$ & $0.566 (\pm0.030)$ & - & - & - & - \\
		Kim						& $0.625 (\pm0.019)$ & $0.602 (\pm0.019)$ & - & - & - & - \\
		Dilated Kim				& $0.675 (\pm0.049)$ & $0.635 (\pm0.049)$ & - & - & - & - \\
		\hline
		\bottomrule
	\end{tabular}
\end{table}		% Evaluation
% A chapter named 'Your first document' is created
\chapter{Conclusion} \label{chapter:conclusion}

Here you have only the put all the concept that could be presented in the work and then in the literature review you will illustrate different proposed solution to the exposed topics.		% Conclusions


% The bibliography is printed with \bibliography{}. With the command \bibliographystyle{} a style is picked.
%\bibliographystyle{plain}
%\bibliography{refs/references}


% To close your document, add the \end{document} command. Everything after this command will not be processed.
\end{document}